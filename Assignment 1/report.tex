% Page Setup -----------------------------------------------------
\documentclass[11pt, a4paper, fleqn, oneside]{article}
\usepackage[a4paper]{geometry}
\geometry{verbose,tmargin=1in,bmargin=1in,lmargin=.6in,rmargin=.6in}
% Packages ------------------------------------------------------
\usepackage{amsmath}
\usepackage{amsfonts}
\usepackage{hyperref}
\usepackage{lastpage}
\usepackage{setspace}
\usepackage{graphicx}
\usepackage{enumitem}
\setlist[description]{leftmargin=\parindent,labelindent=\parindent}
\usepackage{multicol}
\raggedcolumns
% Head / Foot ----------------------------------------------------
\usepackage{fancyhdr}
\setlength{\headheight}{15.2pt}
\pagestyle{fancy}
\lhead{}
\chead{CSCI3180 Principles of Programming Languages, Spring 2015-2016\\}
\rhead{Assignment 1 Report}
\rfoot{\today}
\cfoot{ }
\lfoot{Page \thepage\ of \pageref{LastPage}}

\begin{document}
\subsection*{Language Review (versus C)}
\subsubsection*{Data I/O}
In FORTRAN, data I/O is very similar to C in a sense that the console acts like a file and
it assigns an integer file-descriptor to each opened file. You can read / write to files using
a predefined format string like libc's printf / scanf.\\
In COBOL, the console is separate from files and the files that the program uses must be defined
in a special header along with the contained data format.
\subsubsection*{Arithmetic Operations}
In FORTRAN arithmetic operations is similar to C, ASCII maths expressions can be evaluated
in-line and data types are also similar, some operators are replaced by words due to FORTRAN's
small character set but they almost behave the same.\\
In COBOL, all data types seem to be strings with 3 sets: [0-9], [a-zA-Z], and *.
Arithmetic operations resembles that of the English language although ASCII maths expressions
can also be evaluated via the COMPUTE statement.
\subsubsection*{Subroutine}
Both FORTRAN and COBOL offers code organization features with subroutines.
FORTRAN has C-like functions where arguments are passed by reference while
all variables(fields) in COBOL are global.

\subsection*{DDA Implementation}
The hardest part about this task is not in fact the use of ancient programming languages
because they are very much full-featured and productions-ready, however the limitation of using
only IF and GO TO requires some tricks to emulate higher-level constructs such as for-loops.
\subsubsection*{FORTRAN}
Loops and branches can be easily implemented using labels and conditional jumps, techniques are
very similar to those of assembly programming.
\subsubsection*{COBOL}
Implementation in COBOL was not trivial because the language did not have labels. Instead,
subroutine names are used with the GO TO statement, this will disturb the subroutine call
trace and thus the executions flow as well if not used properly. (i.e. A subroutine will not
return if GO TO is used to direct control elsewhere). The workaround is to have a subroutine
conditionally GO TO itself to implement a loop, since the GO TO statement will not add to
the call trace, this sill simply result in a flat loop.

\end{document}
